% =============================================================================
% AI Coding Tools Market Research - Report Structure Overview
% =============================================================================

\section{Report Overview}

\begin{frame}{Report Overview}
\justifying
\textbf{Contents}

\vspace{8pt}
\textbf{Part I: Market Landscape}
\begin{itemize}\tightlist
  \item Market 1: Foundation Models for Software Engineering
  \item Market 2: General IDEs/Extensions/CLI Agents for Software Engineering
  \item Market 3: Specialized Agents/Tools for Software Engineering
  \item Market 4: Vibe Coding Tools
\end{itemize}

\vspace{8pt}
\textbf{Part II: Research Questions}
\begin{itemize}\tightlist
  \item RQ1: Economic Impact Prediction
  \item RQ2: Competitive Dynamics Prediction
\end{itemize}
\end{frame}

% =============================================================================
% PART I: CURRENT MARKET LANDSCAPE
% =============================================================================

\subsection{Market Landscape Overview}

\begin{frame}{Part I: Market Landscape Overview}
\justifying
\textbf{Objective:} Analyze the current state of 4 interconnected markets in the AI coding tool ecosystem, and discovery trends for the \textcolor{red}{next 6-12 months}. 

\vspace{10pt}
\begin{enumerate}\tightlist
  \item \textbf{Market 1: Foundation Models for Software Engineering}\\
  Foundation models (closed/open) used as backbones for coding and software engineering tasks.
  
  \item \textbf{Market 2: General IDEs/CLIs/Coding Agents}\\
  Developer tools that integrate foundation models: IDE extensions, CLI tools, and coding agents.
  
  \item \textbf{Market 3: Specialized Agents/Tools for Software Engineering}\\
  Purpose-built agents/tools for specific software engineering stages: testing, code review, documentation, DevOps, etc.
  
  \item \textbf{Market 4: Vibe Coding Tools}\\
  Natural language-to-application tools enabling non-programmers to build software through conversation.
\end{enumerate}
\end{frame}

\begin{frame}{Part I: Market Landscape Executive Summary}
\justifying
\small

\textbf{Market 1: Foundation Models} --- Top 3 proprietary models (Claude, GPT, Gemini) score \textcolor{red}{71.8--74.4\%} on SWE-Bench Verified, creating an \textcolor{red}{8.4-point capability gap} vs open-weight (55--63\%) that widens to \textcolor{red}{22+ points} on agentic benchmarks like Terminal-Bench 2.0. Performance degradation on SWE-Bench Pro reveals generalization weakness: Claude drops \textcolor{red}{-38.3\%}, indicating reliance on memorized GitHub patterns.

\vspace{5pt}
\textbf{Market 2: IDE/CLI/Agents} --- \textcolor{red}{Distribution dominates capability}: VS Code/Visual Studio (\textcolor{red}{50M developers}) dwarfs AI-natives (Cursor \textasciitilde1M, Windsurf \textasciitilde800k, TRAE >1M). GitHub Copilot's \textcolor{red}{64.2M installs} exceeds competitors by >20x, but multi-model adoption signals commoditization. Claude Code CLI leads at \textcolor{red}{6.25M weekly downloads} (12x Codex), with MCP creating network effects beyond model performance.

\vspace{5pt}
\textbf{Market 3: Specialized Tools} --- 6 sub-markets with combined TAM of \textcolor{red}{\$68--127B}. Security (\$33.7B $\rightarrow$ \$55B by 2029) and DevOps have \textbf{strongest moats}; Code Review \textbf{most vulnerable} to commoditization. M\&A accelerating: \textcolor{red}{39\% increase in 2025} to \$4.3T globally.

\vspace{5pt}
\textbf{Market 4: Vibe Coding} --- Consumer-style velocity: Lovable \textcolor{red}{>\$200M ARR}, Retool \textasciitilde\$120M, Replit \textasciitilde\$106M. Two winner profiles emerge: high-ARPPU prosumer (Lovable \textasciitilde\$1,111/year) vs high-volume consumer (Bolt 5M users at \textasciitilde\$8 ARPU). \textbf{Reality check:} \textcolor{red}{80--90\%} prototypes, not production; TAM ceiling \textcolor{red}{\$5--15B}.
\end{frame}


\subsection{Research Questions Overview}

% -----------------------------------------------------------------------------
% RQ1: Economic Impact
% -----------------------------------------------------------------------------

\begin{frame}{Part II: RQ1 Economic Impact Overview}
\justifying
\textbf{Research Question:} What will be the economic impact of AI coding tools on the software engineering industry \textcolor{red}{5 years from now in 2030}?


\vspace{6pt}
\textbf{Sub-RQs:}
\begin{enumerate}\tightlist
  \item \textbf{RQ1.1} Impact on Big Tech (headcount, delivery speed, role shifts)
  \item \textbf{RQ1.2} Impact on Traditional Industry IT Departments (finance, semiconductor, automobile, healthcare, government)
  \item \textbf{RQ1.3} Impact on Startup Ecosystem \& VC Market
  \item \textbf{RQ1.4} Impact on Labor Market (job displacement, new roles, wage effects)
  \item \textbf{RQ1.5} Impact on Macro Economic Growth (GDP contribution, productivity paradox)
  \item \textbf{RQ1.6} Negative Externalities \& Risks
\end{enumerate}
\end{frame}

\begin{frame}{Part II: RQ1 Economic Impact Executive Summary}
\justifying
\small
\begin{itemize}
  \item \textbf{RQ1.1 Big Tech:} Per 10,000-engineer org at \$320K fully-loaded: Conservative \textcolor{red}{\$50--80M/year} (2--3\% capacity gain); Optimistic \textcolor{red}{\$250--300M/year} (8--9\% gain). Engineer:PM ratio shifts from 8:1 $\rightarrow$ \textcolor{red}{5--6:1}. Expect \textcolor{red}{>70\%} Big Tech penetration by 2027.

  \item \textbf{RQ1.2 Traditional Industries:} Finance (5K eng): \textcolor{red}{\$12--75M/year} but NIST compliance adds 15--20\% overhead. Semiconductor: \textcolor{red}{\$8--15M/year} with \textcolor{red}{high downside risk} (P10: -\$3M)---proprietary HDL limits exposure. Healthcare/Government: near breakeven due to HIPAA/air-gap constraints.

  \item \textbf{RQ1.3 Startups/VC:} MVP timelines: \textcolor{red}{12 weeks $\rightarrow$ 3--4 weeks}. Minimum viable team: \textcolor{red}{8--12 $\rightarrow$ 2--4 engineers}. \textbf{Catch:} Lower barriers = more competitors = \textcolor{red}{margin compression}. Defensibility shifts from ``we built it'' to distribution/data/network effects.

  \item \textbf{RQ1.4--1.6 Labor \& Macro:} BLS 25\% growth $\rightarrow$ likely \textcolor{red}{10--15\%} with AI. \textbf{High risk:} junior devs, manual QA. \textbf{Low risk:} architects, ML, security. Macro: \textcolor{red}{0.08--0.45\% GDP boost}---meaningful but not transformative. \textbf{DORA 2024 paradox:} AI adoption correlates with \textcolor{red}{worse} throughput (-1.5\%) and stability (-7.2\%) at org level.
\end{itemize}
\end{frame}


% -----------------------------------------------------------------------------
% RQ2: Competitive Dynamics
% -----------------------------------------------------------------------------

\begin{frame}{Part II: RQ2 Competitive Dynamics Overview}
\justifying
\textbf{Research Question:} What will the competitive dynamics of the AI coding ecosystem look like \textcolor{red}{5 years from now in 2030?}

\vspace{6pt}
\textbf{Sub-RQs:}
\begin{enumerate}\tightlist
  \item \textbf{RQ2.1} Intra-Market Dynamics (each market's internal structure)
  \item \textbf{RQ2.2} Inter-Market Competition (can players invade adjacent markets?)
  \item \textbf{RQ2.3} Key Player Deep Dives (Anthropic, Cursor, Microsoft, etc.)
  \item \textbf{RQ2.4} Moats \& Defensibility Analysis
  \item \textbf{RQ2.5} Emerging Opportunities \& White Spaces
  \item \textbf{RQ2.6} Wildcards \& Disruption Scenarios
\end{enumerate}
\end{frame}

\begin{frame}{Part II: RQ2 Competitive Dynamics Executive Summary}
\justifying
\small
\begin{itemize}
  \item \textbf{RQ2.1 Market Structure by 2030:} M1 $\rightarrow$ \textcolor{red}{3+2 oligopoly} (3 frontier + Western open + Chinese); \$100M+ training cost creates natural oligopoly. M2 $\rightarrow$ \textcolor{red}{2--3 dominant IDEs}; 60--70\% acquired/shut. M3 $\rightarrow$ \textcolor{red}{60\%+ absorbed}; Security (25\% consolidated) survives, Code Review (80\%) doesn't. M4 $\rightarrow$ \textcolor{red}{\$5--15B TAM}, not \$100B+ fantasy.

  \item \textbf{RQ2.2--2.3 Key Players:} \textbf{Anthropic}---best vertical position (Terminal-Bench \textcolor{red}{57.8\%}, CLI \textcolor{red}{6.25M/week}, MCP lock-in); gap: IDE distribution (80x smaller than Copilot). \textbf{Microsoft}---50M MAU distribution but OpenAI dependency uncertain post-2030. \textbf{Cursor}---Tab model \textcolor{red}{28\% higher accept rate} but \textcolor{red}{12--18mo lead narrowing}; binary outcome: \$8--15B acquisition or 5M+ breakout. \textbf{Google}---5x context window, 3B Workspace users underutilized.

  \item \textbf{RQ2.4 Moat Test (must pass 2+/4):} (1) Not replicable in 18--24mo? (2) Data flywheel? (3) Distribution chokepoint? (4) Governance lock-in? \textbf{Ranking:} \textcolor{red}{M1 (4/4) $>$ Security (3/4) $>$ M2 (2/4) $>$ M4 (0--1/4)}. Open-weight lacks production feedback---explaining persistent \textcolor{red}{22+ point Terminal-Bench gap}.

  \item \textbf{RQ2.5--2.6 Opportunities \& Wildcards:} Agent eval/policy (\textcolor{red}{\$2--5B}), observability (\textcolor{red}{\$1--3B}). \textbf{Wildcards:} Security incident (\textcolor{red}{40--60\%})---12--24mo slowdown if major; IP litigation (\textcolor{red}{20--35\%}); China decoupling (\textcolor{red}{50--70\%})---lose 20--30\% global TAM.
\end{itemize}
\end{frame}


