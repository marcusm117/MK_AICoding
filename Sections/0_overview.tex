% =============================================================================
% AI Coding Tools Market Research - Report Structure Overview
% =============================================================================

\section{Report Overview}

\begin{frame}{Report Overview}
\justifying
\textbf{Contents}

\vspace{8pt}
\textbf{Part I: Market Landscape}
\begin{itemize}\tightlist
  \item Market 1: Foundation Models for Software Engineering
  \item Market 2: General IDEs/CLIs/Coding Agents
  \item Market 3: Specialized Agents/Tools for Software Engineering
  \item Market 4: Vibe Coding Tools
\end{itemize}

\vspace{8pt}
\textbf{Part II: Research Questions}
\begin{itemize}\tightlist
  \item RQ1: Economic Impact Prediction
  \item RQ2: Competitive Dynamics Prediction
\end{itemize}
\end{frame}

% =============================================================================
% PART I: CURRENT MARKET LANDSCAPE
% =============================================================================

\part{Market Landscape}

\begin{frame}{Part I: Market Landscape Overview}
\justifying
\textbf{Objective:} Analyze the current state of 4 interconnected markets in the AI coding tool ecosystem, and discovery trends for the \textcolor{red}{next 6-12 months}. 

\vspace{10pt}
\begin{enumerate}\tightlist
  \item \textbf{Market 1: Foundation Models for Software Engineering}\\
  Foundation models (closed/open) used as backbones for coding and software engineering tasks.
  
  \item \textbf{Market 2: General IDEs/CLIs/Coding Agents}\\
  Developer tools that integrate foundation models: IDE extensions, CLI tools, and coding agents.
  
  \item \textbf{Market 3: Specialized Agents/Tools for Software Engineering}\\
  Purpose-built agents/tools for specific software engineering stages: testing, code review, documentation, DevOps, etc.
  
  \item \textbf{Market 4: Vibe Coding Tools}\\
  Natural language-to-application tools enabling non-programmers to build software through conversation.
\end{enumerate}
\end{frame}

\begin{frame}{Part I: Market Landscape Executive Summary}
\justifying
\small

\textbf{Market 1: Foundation Models} — Frontier closed-source models (Claude Opus 4.5, GPT-5.2, Gemini 3 Pro) lead on SWE-bench Verified ($\sim$75--81\%), but real-world transfer (SWE-bench Pro) drops to $\sim$45\%. Open-weight models lag $\sim$15--20pt but offer deployment flexibility.

\vspace{6pt}
\textbf{Market 2: IDE/CLI/Agents} — Cursor, Windsurf, and Claude Code dominate developer mindshare. Key differentiator: multi-model routing + context management. GitHub Copilot faces disruption from agentic competitors.

\vspace{6pt}
\textbf{Market 3: Specialized SE Tools} — Fragmented market with point solutions for testing (Codium), code review (CodeRabbit), and DevOps. Early consolidation signals as IDE vendors add specialized features.

\vspace{6pt}
\textbf{Market 4: Vibe Coding} — Emerging category led by Bolt, Lovable, and v0. Enables non-developers to ship production apps. Growth constrained by complexity ceiling and maintenance challenges.

\vspace{6pt}
\textit{Cross-cutting trend: Vertical integration (model $\rightarrow$ tool $\rightarrow$ platform) accelerating across all markets.}
\end{frame}

% =============================================================================
% PART II: RESEARCH QUESTIONS
% =============================================================================

\part{Research Questions}

% -----------------------------------------------------------------------------
% RQ1: Economic Impact
% -----------------------------------------------------------------------------

\begin{frame}{Part II: RQ1 Economic Impact Overview}
\justifying
\textbf{Research Question:} What will be the economic impact of AI coding tools on the software engineering industry \textcolor{red}{5 years from now in 2030}?


\vspace{6pt}
\textbf{Sub-RQs:}
\begin{enumerate}\tightlist
  \item \textbf{RQ1.1} Impact on Big Tech (headcount, delivery speed, role shifts)
  \item \textbf{RQ1.2} Impact on Traditional Industry IT Departments (finance, semiconductor, automobile, healthcare, government)
  \item \textbf{RQ1.3} Impact on Startup Ecosystem \& VC Market
  \item \textbf{RQ1.4} Impact on Labor Market (job displacement, new roles, wage effects)
  \item \textbf{RQ1.5} Impact on Macro Economic Growth (GDP contribution, productivity paradox)
  \item \textbf{RQ1.6} Negative Externalities \& Risks
\end{enumerate}
\end{frame}

\begin{frame}{Part II: RQ1 Economic Impact Executive Summary}
\justifying
\small
\begin{itemize}
  \item \textbf{RQ1.1 Big Tech:} Productivity gains of \textcolor{red}{15--30\%} are projected to generate \textcolor{red}{\$50--100B} in annual savings across FAANG+ by 2028, with engineer-to-PM ratios shifting from 8:1 to approximately 5:1 as AI augments individual developer output.

  \item \textbf{RQ1.2 Traditional Industries:} Financial services IT departments anticipate 10--20\% cost reductions (\$20--40B globally), while semiconductor/EDA sectors face limited near-term impact (5--10\%) due to proprietary codebase complexity and domain-specific toolchains.

  \item \textbf{RQ1.3 Startups/VC:} MVP development timelines are compressing from \textcolor{red}{12 weeks to 3--4 weeks}, enabling 2-person teams to deliver output previously requiring 8--10 engineers, with VCs funding 20--30\% smaller teams at equivalent valuations.

  \item \textbf{RQ1.4--1.6 Labor \& Macro:} Net job growth remains likely (demand elasticity $>$ displacement), though junior roles face highest risk while architect/ML positions remain insulated. Macro impact: \textcolor{red}{0.2--0.5\% GDP contribution}. Key risks include security vulnerabilities, IP litigation, and vendor concentration.
\end{itemize}
\end{frame}


% -----------------------------------------------------------------------------
% RQ2: Competitive Dynamics
% -----------------------------------------------------------------------------

\begin{frame}{Part II: RQ2 Competitive Dynamics Overview}
\justifying
\textbf{Research Question:} What will the competitive dynamics of the AI coding ecosystem look like \textcolor{red}{5 years from now in 2030?}

\vspace{6pt}
\textbf{Sub-RQs:}
\begin{enumerate}\tightlist
  \item \textbf{RQ2.1} Intra-Market Dynamics (each market's internal structure)
  \item \textbf{RQ2.2} Inter-Market Competition (can players invade adjacent markets?)
  \item \textbf{RQ2.3} Key Player Deep Dives (Anthropic, Cursor, Microsoft, etc.)
  \item \textbf{RQ2.4} Moats \& Defensibility Analysis
  \item \textbf{RQ2.5} Emerging Opportunities \& White Spaces
  \item \textbf{RQ2.6} Wildcards \& Disruption Scenarios
\end{enumerate}
\end{frame}

\begin{frame}{Part II: RQ2 Competitive Dynamics Executive Summary}
\justifying
\small
\begin{itemize}
  \item \textbf{RQ2.1 Intra-Market Consolidation:} M1 consolidates to \textcolor{red}{3--4 frontier providers} + open-weight tier; M2 reduces to 2--3 dominant IDEs; M3 sees \textcolor{red}{60\%+ acquired}; M4 remains fragmented with niche positioning.

  \item \textbf{RQ2.2 Inter-Market Dynamics:} Foundation model companies (\textbf{Anthropic}, \textbf{OpenAI}) are projected to capture \textcolor{red}{30--40\%} of M2 revenue via vertical integration, while IDE-native players like \textbf{Cursor} are unlikely to build competitive models---their moat is UX/distribution, not capability.

  \item \textbf{RQ2.3--2.4 Key Players \& Moats:} \textbf{Microsoft} faces strategic inflection (post-OpenAI plan needed); \textbf{Anthropic} best positioned for vertical integration; \textbf{Google} underperforms its distribution assets. Moat durability: M1 (compute/data) $>$ M2 (UX/habit) $>$ M3 (domain expertise) $>$ M4 (community).

  \item \textbf{RQ2.5--2.6 Opportunities \& Wildcards:} Emerging opportunities include AI-native dev paradigms and enterprise platforms. \textcolor{red}{Wildcard scenarios:} AGI-level coding, major security incidents, IP litigation waves, and China ecosystem decoupling.
\end{itemize}
\end{frame}
