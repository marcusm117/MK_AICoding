% =============================================================================
% Part I / Market 3: Specialized Agents/Tools for SE
% =============================================================================

\section{Market 3: Specialized Agents/Tools for SE}

\begin{frame}{Market 3: Scope \& Definition}
\justifying
\small
\begin{itemize}
  \item \textbf{Research Object:} Purpose-built AI agents and tools optimized for specific software engineering stages---distinct from general-purpose IDEs/agents (Market 2) that attempt broad coverage.

  \item \textbf{Six Sub-Markets:}
  \begin{enumerate}
    \item \textbf{Documentation / Knowledge:} DeepWiki, Mintlify, Swimm, ReadMe, GitBook, Notion AI.
    \item \textbf{Spec / Architecture / Design:} Figma Dev Mode, Postman, Stoplight, Eraser, v0, Fern.
    \item \textbf{Testing / QA:} Diffblue, Momentic, Mabl, Applitools, BrowserStack, Launchable.
    \item \textbf{Code Review / PR Workflow:} CodeRabbit, Qodo, Graphite, SonarQube, CodeScene, LinearB.
    \item \textbf{DevOps / Infrastructure:} Pulumi AI, env0, Spacelift, Kubiya, Port, Backstage.
    \item \textbf{Security / Compliance:} Snyk, Semgrep, Socket, GitGuardian, Vanta, Drata.
  \end{enumerate}

  \item \textbf{Core Assessment Dimensions:} Three dimensions determine whether a vertical can sustain stand-alone vendors:
  \begin{enumerate}
    \item \textbf{Market Size} $\rightarrow$ TAM for each vertical---is the segment large enough to support dedicated vendors?
    \item \textbf{Technical Moat} $\rightarrow$ what domain-specific capabilities cannot be easily replicated by Market 2 players?
    \item \textbf{AI Potential} $\rightarrow$ what percentage of the workflow can AI fully automate vs. requiring human judgment?
  \end{enumerate}
\end{itemize}
\end{frame}

% =============================================================================
% Sub-Market 1: Documentation / Knowledge / Onboarding
% =============================================================================

\begin{frame}{Market 3.1: Documentation / Knowledge---Key Players}
% Table: Market 3 - Documentation / Knowledge / Onboarding
% File: Tables/P1M3-documentation.tex

\begin{table}[H]
\centering
\caption{Documentation \& Knowledge Tools---Key Players by Traction}
\label{tab:m3-documentation}

\small
\resizebox{\textwidth}{!}{%
\begin{tabular}{@{}rllllr@{}}
\toprule
\textbf{Rank} & \textbf{Company} & \textbf{Product} & \textbf{Sub-Category} & \textbf{Best Traction Proxy} & \textbf{Funding} \\
\midrule

1 & Atlassian & \citelink{https://www.atlassian.com/software/confluence}{Confluence AI} & Internal KB / Wikis & \citelink{https://www.atlassian.com/company}{300k+ customers (platform)} & Public \\
\addlinespace[2pt]

2 & Notion Labs & \citelink{https://www.notion.so/product/ai}{Notion AI} & Internal KB / Docs & \citelink{https://www.notion.so/blog/100-million-users}{100M+ users (platform)} & \$2.75B \\
\addlinespace[2pt]

3 & GitBook & \citelink{https://www.gitbook.com/}{GitBook} & Developer Docs & \citelink{https://www.gitbook.com/}{1M+ docs published} & \$17M \\
\addlinespace[2pt]

4 & ReadMe & \citelink{https://readme.com/}{ReadMe} & API Docs / Portal & \citelink{https://readme.com/customers}{6,000+ companies} & \$39M \\
\addlinespace[2pt]

5 & Mintlify & \citelink{https://mintlify.com/}{Mintlify} & Code-to-Docs & \citelink{https://mintlify.com/customers}{2,500+ companies} & \$22.5M \\
\addlinespace[2pt]

6 & Swimm & \citelink{https://swimm.io/}{Swimm} & Auto-Sync Docs & \citelink{https://swimm.io/customers}{500+ teams} & \$28M \\
\addlinespace[2pt]

7 & Cognition & \citelink{https://cognition.ai/blog/deepwiki}{DeepWiki} & Repo Wiki & \citelink{https://deepwiki.com/}{1M+ repos indexed} & \$175M (Devin) \\
\addlinespace[2pt]

8 & Meta (Open-Source) & \citelink{https://docusaurus.io/}{Docusaurus} & Static Docs & \citelink{https://github.com/facebook/docusaurus}{58k GitHub stars} & Open-Source \\

\bottomrule
\end{tabular}%
}

\vspace{4pt}
\raggedright
\scriptsize
\textit{Notes:} Rankings prioritize paid customer counts where available; platform metrics (Confluence, Notion) reflect total platform adoption, not AI feature usage specifically.

\end{table}

\end{frame}

\begin{frame}{Market 3.1: Documentation / Knowledge---Analysis}
\justifying
\small
\begin{itemize}
  \item \textbf{Market Size (\$2--4B TAM):} Developer documentation tools represent a \textcolor{red}{\$2--4B} addressable market including API docs, internal wikis, and code-to-doc automation. \citelink{https://readme.com/}{ReadMe} (6,000+ companies), \citelink{https://mintlify.com/}{Mintlify} (2,500+ companies), and \citelink{https://www.gitbook.com/}{GitBook} (1M+ docs) have carved meaningful niches. However, \citelink{https://www.atlassian.com/software/confluence}{Confluence} (300k+ customers) and \citelink{https://www.notion.so/}{Notion} (100M+ users) dominate the broader knowledge management space---documentation specialists must differentiate on developer-specific workflows.

  \item \textbf{Technical Moat (Medium):} Domain-specific moats include: (1) \textbf{code-doc synchronization}---\citelink{https://swimm.io/}{Swimm} auto-updates docs when code changes; (2) \textbf{API schema integration}---ReadMe ingests OpenAPI specs natively; (3) \textbf{repo-level context}---\citelink{https://cognition.ai/blog/deepwiki}{DeepWiki} builds architecture wikis from codebase analysis. General agents (Market 2) struggle here because documentation quality requires \textbf{persistent codebase state}, not one-shot generation. The moat is \textbf{integration depth}, not AI capability per se.

  \item \textbf{AI Potential (High---70--90\%):} Documentation is among the \textcolor{red}{highest AI automation potential} verticals. \citelink{https://mintlify.com/}{Mintlify} generates docs from code comments; \citelink{https://cognition.ai/blog/deepwiki}{DeepWiki} produces repo architecture explainers autonomously. Remaining human work: \textbf{strategic narrative}, \textbf{tutorial pedagogy}, and \textbf{accuracy verification}. Near-term ceiling is maintaining context across large monorepos with complex interdependencies.
\end{itemize}
\end{frame}

% =============================================================================
% Sub-Market 2: Spec / Architecture / Design
% =============================================================================

\begin{frame}{Market 3.2: Spec / Architecture / Design---Key Players}
% Table: Market 3 - Spec / Architecture / Design (Pre-Coding)
% File: Tables/P1M3-spec-design.tex

\begin{table}[H]
\centering
\caption{Spec / Architecture / Design Tools---Key Players by Traction}
\label{tab:m3-spec-design}

\small
\resizebox{\textwidth}{!}{%
\begin{tabular}{@{}rllllr@{}}
\toprule
\textbf{Rank} & \textbf{Company} & \textbf{Product} & \textbf{Sub-Category} & \textbf{Best Traction Proxy} & \textbf{Funding} \\
\midrule

1 & Figma & \citelink{https://www.figma.com/}{Figma Dev Mode} & Design-to-Code & \citelink{https://www.figma.com/blog/figma-4-million-users/}{4M+ users} & Acquired (\$20B) \\
\addlinespace[2pt]

2 & Miro & \citelink{https://miro.com/}{Miro AI} & Diagramming & \citelink{https://miro.com/about/}{70M+ users} & \$1.7B \\
\addlinespace[2pt]

3 & Lucid Software & \citelink{https://www.lucidchart.com/}{Lucidchart} & System Diagrams & \citelink{https://www.lucidchart.com/pages/}{70M+ users} & \$500M+ \\
\addlinespace[2pt]

4 & Postman & \citelink{https://www.postman.com/}{Postman} & API Design & \citelink{https://www.postman.com/company/about-postman/}{30M+ developers} & \$433M \\
\addlinespace[2pt]

5 & SmartBear & \citelink{https://swagger.io/tools/swaggerhub/}{SwaggerHub} & API Contract & \citelink{https://smartbear.com/company/about-us/}{16M+ developers (platform)} & Private \\
\addlinespace[2pt]

6 & Stoplight & \citelink{https://stoplight.io/}{Stoplight} & API Governance & \citelink{https://stoplight.io/customers}{1,000+ companies} & \$45M \\
\addlinespace[2pt]

7 & Vercel & \citelink{https://v0.dev/}{v0} & UI Generation & \citelink{https://v0.dev/}{500k+ generations/month} & \$563M (Vercel) \\
\addlinespace[2pt]

8 & Eraser & \citelink{https://eraser.io/}{Eraser AI} & Tech Diagrams & \citelink{https://eraser.io/}{50k+ teams} & \$14M \\
\addlinespace[2pt]

9 & Fern & \citelink{https://buildwithfern.com/}{Fern} & SDK Generation & \citelink{https://buildwithfern.com/customers}{200+ companies} & \$32M \\
\addlinespace[2pt]

10 & Redocly & \citelink{https://redocly.com/}{Redocly} & API Docs & \citelink{https://redocly.com/customers}{1,500+ companies} & \$20M \\

\bottomrule
\end{tabular}%
}

\vspace{4pt}
\raggedright
\scriptsize
\textit{Notes:} Rankings prioritize developer/user counts. Figma, Miro, Lucidchart are design-first platforms with developer workflow features; Postman, Stoplight, Fern are API-first.

\end{table}

\end{frame}

\begin{frame}{Market 3.2: Spec / Architecture / Design---Analysis}
\justifying
\small
\begin{itemize}
  \item \textbf{Market Size (\$5--8B TAM):} The combined market for API design (\citelink{https://www.postman.com/}{Postman} 30M+ developers), diagramming (\citelink{https://miro.com/}{Miro} 70M+ users, \citelink{https://www.lucidchart.com/}{Lucidchart} 70M+ users), and design-to-code (\citelink{https://www.figma.com/}{Figma} 4M+ users) represents \textcolor{red}{\$5--8B TAM}. This is a \textbf{fragmented market}---no single vendor owns the full ``spec-to-code'' workflow. Sub-segments (API-first, diagram-first, design-first) have distinct buyers and workflows.

  \item \textbf{Technical Moat (Medium--High):} Key moats: (1) \textbf{API contract governance}---\citelink{https://stoplight.io/}{Stoplight} and \citelink{https://swagger.io/tools/swaggerhub/}{SwaggerHub} enforce OpenAPI compliance across teams; (2) \textbf{design system alignment}---\citelink{https://www.figma.com/}{Figma Dev Mode} connects designs to component libraries; (3) \textbf{SDK generation}---\citelink{https://buildwithfern.com/}{Fern} auto-generates client libraries from API specs. Market 2 agents lack the \textbf{schema-aware tooling} and \textbf{multi-stakeholder governance workflows} these specialists provide.

  \item \textbf{AI Potential (Medium---50--70\%):} AI can generate diagrams from text (\citelink{https://eraser.io/}{Eraser AI}), produce UI components from prompts (\citelink{https://v0.dev/}{v0}), and draft API specs. However, \textbf{architecture decisions require human judgment}---system boundaries, scaling tradeoffs, and security implications resist full automation. The ``pre-coding'' phase is where \textbf{intent formation} happens; AI assists but doesn't replace the architect's role in complex systems.
\end{itemize}
\end{frame}

% =============================================================================
% Sub-Market 3: Testing / QA
% =============================================================================

\begin{frame}{Market 3.3: Testing / QA---Key Players}
% Table: Market 3 - Testing / QA
% File: Tables/P1M3-testing.tex

\begin{table}[H]
\centering
\caption{Testing \& QA Tools---Key Players by Traction}
\label{tab:m3-testing}

\small
\resizebox{\textwidth}{!}{%
\begin{tabular}{@{}rllllr@{}}
\toprule
\textbf{Rank} & \textbf{Company} & \textbf{Product} & \textbf{Sub-Category} & \textbf{Best Traction Proxy} & \textbf{Funding} \\
\midrule

1 & Tricentis & \citelink{https://www.tricentis.com/}{Tricentis} & Enterprise Testing & \citelink{https://www.tricentis.com/company}{2,400+ enterprise customers} & \$1.7B (acq.) \\
\addlinespace[2pt]

2 & BrowserStack & \citelink{https://www.browserstack.com/}{BrowserStack} & Test Platform & \citelink{https://www.browserstack.com/customers}{50k+ customers} & \$450M \\
\addlinespace[2pt]

3 & LambdaTest & \citelink{https://www.lambdatest.com/}{LambdaTest} & Cloud Testing & \citelink{https://www.lambdatest.com/customers}{15k+ customers} & \$110M \\
\addlinespace[2pt]

4 & Applitools & \citelink{https://applitools.com/}{Applitools} & Visual Testing & \citelink{https://applitools.com/customers}{800+ enterprise customers} & \$120M \\
\addlinespace[2pt]

5 & Mabl & \citelink{https://www.mabl.com/}{Mabl} & AI E2E Testing & \citelink{https://www.mabl.com/customers}{500+ customers} & \$80M \\
\addlinespace[2pt]

6 & Testim (Tricentis) & \citelink{https://www.testim.io/}{Testim} & AI Test Authoring & \citelink{https://www.testim.io/customers}{1,000+ customers} & Acquired \\
\addlinespace[2pt]

7 & Diffblue & \citelink{https://www.diffblue.com/}{Diffblue Cover} & Unit Test Gen (Java) & \citelink{https://www.diffblue.com/customers}{100+ enterprise customers} & \$35M \\
\addlinespace[2pt]

8 & Momentic & \citelink{https://momentic.ai/}{Momentic} & AI E2E Agent & \citelink{https://momentic.ai/}{Early stage} & \$6M \\
\addlinespace[2pt]

9 & Launchable & \citelink{https://www.launchableinc.com/}{Launchable} & Test Intelligence & \citelink{https://www.launchableinc.com/customers}{100+ customers} & \$20M \\

\bottomrule
\end{tabular}%
}

\vspace{4pt}
\raggedright
\scriptsize
\textit{Notes:} Rankings prioritize customer counts. Tricentis, BrowserStack, LambdaTest are platform plays; Diffblue, Momentic, Launchable are AI-native specialists.

\end{table}

\end{frame}

\begin{frame}{Market 3.3: Testing / QA---Analysis}
\justifying
\small
\begin{itemize}
  \item \textbf{Market Size (\$8--12B TAM):} Software testing is a \textcolor{red}{\$8--12B} market spanning unit testing, E2E testing, visual regression, and test infrastructure. \citelink{https://www.tricentis.com/}{Tricentis} (2,400+ enterprise customers, \$1.7B acquisition), \citelink{https://www.browserstack.com/}{BrowserStack} (50k+ customers, \$450M funding), and \citelink{https://applitools.com/}{Applitools} (800+ enterprise customers) demonstrate the market can support multiple large vendors. Testing is a \textbf{proven enterprise budget category}.

  \item \textbf{Technical Moat (High):} Testing tools have the \textbf{strongest technical moats} in Market 3: (1) \textbf{language-specific optimization}---\citelink{https://www.diffblue.com/}{Diffblue} achieves 90\%+ coverage on Java codebases via bytecode analysis, not general LLM inference; (2) \textbf{visual baseline databases}---\citelink{https://applitools.com/}{Applitools} maintains proprietary visual diff algorithms; (3) \textbf{test execution infrastructure}---BrowserStack/LambdaTest provide device farms that general agents cannot replicate. Market 2 players can \textit{generate} tests but cannot \textit{execute} them at scale.

  \item \textbf{AI Potential (High---70--85\%):} Test generation is highly automatable: \citelink{https://www.diffblue.com/}{Diffblue} generates unit tests automatically; \citelink{https://momentic.ai/}{Momentic} uses AI agents to explore and test UIs. The ceiling: \textbf{test oracles}---AI can generate tests but struggles to know \textit{what correct behavior looks like} without human-specified assertions. \citelink{https://www.launchableinc.com/}{Launchable's} test intelligence (flaky detection, test selection) shows AI's near-term value is \textbf{optimizing existing test suites}, not replacing human test design entirely.
\end{itemize}
\end{frame}

% =============================================================================
% Sub-Market 4: Code Review / PR Workflow
% =============================================================================

\begin{frame}{Market 3.4: Code Review / PR Workflow---Key Players}
% Table: Market 3 - Code Review / PR Workflow / Change Intelligence
% File: Tables/P1M3-code-review.tex

\begin{table}[H]
\centering
\caption{Code Review \& PR Workflow Tools---Key Players by Traction}
\label{tab:m3-code-review}
\vspace{-8pt}
\small
\resizebox{\textwidth}{!}{%
\begin{tabular}{@{}rllllr@{}}
\toprule
\textbf{Rank} & \textbf{Company} & \textbf{Product} & \textbf{Sub-Category} & \textbf{Best Traction Proxy} & \textbf{Funding} \\
\midrule

1 & SonarSource & \citelink{https://www.sonarsource.com/}{SonarQube/Cloud} & Code Quality & \citelink{https://techcrunch.com/2022/07/12/sonarqube-sonarsource-raises-412m-for-clean-code/}{7M devs, 400k+ orgs} & \$412M \\
\addlinespace[2pt]

2 & CodeRabbit & \citelink{https://www.coderabbit.ai/}{CodeRabbit} & AI PR Review & \citelink{https://techcrunch.com/2025/09/16/coderabbit-raises-60m-valuing-the-2-year-old-ai-code-review-startup-at-550m/}{\$550M valuation, 30k+ repos} & \$88M \\
\addlinespace[2pt]

3 & Graphite & \citelink{https://graphite.dev/}{Graphite} & PR Stacking & \citelink{https://techcrunch.com/2024/02/06/graphite-raises-52m-series-b/}{1,000+ teams} & \$72M \\
\addlinespace[2pt]

4 & LinearB & \citelink{https://linearb.io/}{LinearB} & Eng Analytics & \citelink{https://techcrunch.com/2022/06/08/linearb-raises-50m-series-b/}{2,000+ teams} & \$71M \\
\addlinespace[2pt]

5 & Qodo & \citelink{https://www.qodo.ai/}{Qodo} & AI Review + Tests & \citelink{https://marketplace.visualstudio.com/items?itemName=Codium.codium}{700k+ VS Code installs} & \$50M \\
\addlinespace[2pt]

6 & DeepSource & \citelink{https://deepsource.com/}{DeepSource} & Static Analysis & \citelink{https://techcrunch.com/2020/02/25/deepsource-raises-2-6m-seed/}{2,500+ teams} & \$5M \\
\addlinespace[2pt]

7 & Codacy & \citelink{https://www.codacy.com/}{Codacy} & Code Quality & \citelink{https://techcrunch.com/2020/08/26/codacy-raises-15m-series-b/}{1,000+ organizations} & \$20M \\
\addlinespace[2pt]

8 & CodeScene & \citelink{https://codescene.com/}{CodeScene} & Change Intelligence & \citelink{https://techcrunch.com/2022/02/01/codescene-raises-10m-series-a/}{200+ enterprise customers} & \$10M \\
\addlinespace[2pt]

9 & Qodo (Open-Source) & \citelink{https://github.com/Codium-ai/pr-agent}{PR-Agent} & AI PR Review & \citelink{https://github.com/Codium-ai/pr-agent}{6.7k GitHub stars} & Open-Source \\

\bottomrule
\end{tabular}%
}

\vspace{4pt}
\raggedright
\scriptsize
\textit{Notes:} Rankings prioritize funding/valuation. \citelink{https://news.nus.edu.sg/nus-spinoff-tech-autocoderover-acquired-by-sonar/}{Sonar acquired AutoCodeRover} (2025) for agentic code development.

\end{table}

\end{frame}

\begin{frame}{Market 3.4: Code Review / PR Workflow---Analysis}
\justifying
\small
\begin{itemize}
  \item \textbf{Market Size (\$3--5B TAM):} Code quality and review tools represent \textcolor{red}{\$3--5B TAM}. \citelink{https://www.sonarsource.com/}{SonarSource} (400k+ organizations, \$412M funding) dominates code quality gates. Engineering analytics (\citelink{https://linearb.io/}{LinearB} 2,000+ teams, \citelink{https://codescene.com/}{CodeScene} 200+ enterprise) is a growing adjacent category. AI-native review (\citelink{https://www.coderabbit.ai/}{CodeRabbit}, \citelink{https://www.qodo.ai/}{Qodo}) is early-stage but growing rapidly.

  \item \textbf{Technical Moat (Medium):} Code review specialists differentiate via: (1) \textbf{rule libraries}---SonarQube has 5,000+ rules across 30+ languages; (2) \textbf{codebase history}---\citelink{https://codescene.com/}{CodeScene} analyzes git history to identify ``hotspots'' and technical debt patterns; (3) \textbf{PR workflow integration}---\citelink{https://graphite.dev/}{Graphite} enables stacked PRs natively. However, this is the \textbf{most vulnerable Market 3 vertical} to Market 2 cannibalization---\citelink{https://marketplace.visualstudio.com/items?itemName=GitHub.copilot}{Copilot} and \citelink{https://cursor.com}{Cursor} already do inline code suggestions that overlap with review feedback.

  \item \textbf{AI Potential (Medium---50--70\%):} AI can catch obvious bugs, style violations, and security issues. \citelink{https://www.coderabbit.ai/}{CodeRabbit} and \citelink{https://www.qodo.ai/}{Qodo} demonstrate useful PR summarization and automated suggestions. The ceiling: \textbf{architectural review} and \textbf{business logic correctness} require human context. AI augments reviewers but won't replace senior engineer judgment on complex changes. The risk: Market 2 agents may subsume basic review functionality, leaving specialists to compete on \textbf{enterprise compliance} and \textbf{analytics}.
\end{itemize}
\end{frame}

% =============================================================================
% Sub-Market 5: DevOps / CI/CD / Infrastructure
% =============================================================================

\begin{frame}{Market 3.5: DevOps / Infrastructure---Key Players}
% Table: Market 3 - DevOps / CI/CD / Infrastructure
% File: Tables/P1M3-devops.tex

\begin{table}[H]
\centering
\caption{DevOps / Infrastructure Tools---Key Players by Traction}
\label{tab:m3-devops}

\small
\resizebox{\textwidth}{!}{%
\begin{tabular}{@{}rllllr@{}}
\toprule
\textbf{Rank} & \textbf{Company} & \textbf{Product} & \textbf{Sub-Category} & \textbf{Best Traction Proxy} & \textbf{Funding} \\
\midrule

1 & Microsoft & \citelink{https://github.com/features/actions}{GitHub Actions} & CI/CD & \citelink{https://github.blog/news-insights/octoverse/octoverse-2024/}{100M+ developers (platform)} & Platform \\
\addlinespace[2pt]

2 & GitLab & \citelink{https://about.gitlab.com/}{GitLab CI} & CI/CD & \citelink{https://about.gitlab.com/company/}{30M+ users} & Public \\
\addlinespace[2pt]

3 & Harness & \citelink{https://www.harness.io/}{Harness} & CI/CD + AI & \citelink{https://www.harness.io/customers}{1,500+ customers} & \$425M \\
\addlinespace[2pt]

4 & Pulumi & \citelink{https://www.pulumi.com/}{Pulumi / Neo} & IaC + AI Agent & \citelink{https://www.pulumi.com/customers}{2,500+ customers} & \$141M \\
\addlinespace[2pt]

5 & Spacelift & \citelink{https://spacelift.io/}{Spacelift} & IaC Orchestration & \citelink{https://spacelift.io/customers}{500+ customers} & \$65M \\
\addlinespace[2pt]

6 & env0 & \citelink{https://www.env0.com/}{env0} & IaC Automation & \citelink{https://www.env0.com/customers}{400+ customers} & \$53M \\
\addlinespace[2pt]

7 & Humanitec & \citelink{https://humanitec.com/}{Humanitec} & Platform Eng & \citelink{https://humanitec.com/customers}{200+ customers} & \$32M \\
\addlinespace[2pt]

8 & Kubiya & \citelink{https://www.kubiya.ai/}{Kubiya} & DevOps AI Agent & \citelink{https://www.kubiya.ai/}{Early stage} & \$14M \\
\addlinespace[2pt]

9 & Port & \citelink{https://www.getport.io/}{Port} & Dev Portal & \citelink{https://www.getport.io/customers}{300+ customers} & \$38M \\
\addlinespace[2pt]

10 & Open-Source & \citelink{https://backstage.io/}{Backstage} & Dev Portal & \citelink{https://github.com/backstage/backstage}{29k GitHub stars} & Open-Source \\

\bottomrule
\end{tabular}%
}

\vspace{4pt}
\raggedright
\scriptsize
\textit{Notes:} Rankings prioritize customer counts. GitHub Actions, GitLab CI dominate via platform; Pulumi, Spacelift, env0 focus on IaC; Port, Humanitec, Backstage target platform engineering.

\end{table}

\end{frame}

\begin{frame}{Market 3.5: DevOps / Infrastructure---Analysis}
\justifying
\small
\begin{itemize}
  \item \textbf{Market Size (\$15--25B TAM):} DevOps/infrastructure is the \textcolor{red}{largest Market 3 vertical at \$15--25B TAM}. \citelink{https://github.com/features/actions}{GitHub Actions} (100M+ developer platform) and \citelink{https://about.gitlab.com/}{GitLab CI} (30M+ users) dominate CI/CD. IaC specialists (\citelink{https://www.pulumi.com/}{Pulumi} 2,500+ customers, \citelink{https://spacelift.io/}{Spacelift} 500+ customers) and platform engineering tools (\citelink{https://www.getport.io/}{Port}, \citelink{https://humanitec.com/}{Humanitec}) carve out niches. This is a \textbf{proven enterprise infrastructure category}.

  \item \textbf{Technical Moat (High):} DevOps tools have \textbf{deep infrastructure integration moats}: (1) \textbf{cloud provider APIs}---Pulumi, env0, Spacelift integrate with AWS/Azure/GCP resource models; (2) \textbf{state management}---IaC tools track infrastructure drift that general agents cannot reason about; (3) \textbf{RBAC and compliance}---enterprise deployments require audit trails, approval workflows, and policy enforcement. \citelink{https://www.pulumi.com/product/neo/}{Pulumi Neo's} AI agent shows how specialists can layer AI on top of \textbf{infrastructure-native abstractions}.

  \item \textbf{AI Potential (Medium---40--60\%):} AI can generate Terraform/Pulumi code, suggest configurations, and automate runbooks. \citelink{https://www.kubiya.ai/}{Kubiya} demonstrates conversational DevOps automation. However, \textbf{infrastructure changes are high-stakes}---production incidents from AI-generated IaC are costly. Human approval gates will persist. The near-term ceiling is \textbf{blast radius management}: AI-generated infra changes need guardrails that prevent catastrophic failures.
\end{itemize}
\end{frame}

% =============================================================================
% Sub-Market 6: Security / Compliance
% =============================================================================

\begin{frame}{Market 3.6: Security / Compliance---Key Players}
% Table: Market 3 - Security / Compliance
% File: Tables/P1M3-security.tex

\begin{table}[H]
\centering
\caption{Security \& Compliance Tools---Key Players by Traction}
\label{tab:m3-security}

\small
\resizebox{\textwidth}{!}{%
\begin{tabular}{@{}rllllr@{}}
\toprule
\textbf{Rank} & \textbf{Company} & \textbf{Product} & \textbf{Sub-Category} & \textbf{Best Traction Proxy} & \textbf{Funding} \\
\midrule

1 & Snyk & \citelink{https://snyk.io/}{Snyk} & SAST + SCA & \citelink{https://snyk.io/customers/}{3,000+ customers} & \$1.0B \\
\addlinespace[2pt]

2 & Checkmarx & \citelink{https://checkmarx.com/}{Checkmarx} & SAST & \citelink{https://checkmarx.com/company/}{1,800+ enterprise customers} & \$1.15B (acq.) \\
\addlinespace[2pt]

3 & Veracode & \citelink{https://www.veracode.com/}{Veracode} & SAST + DAST & \citelink{https://www.veracode.com/customers}{2,500+ customers} & \$2.5B (acq.) \\
\addlinespace[2pt]

4 & Sonatype & \citelink{https://www.sonatype.com/}{Sonatype} & SCA / Supply Chain & \citelink{https://www.sonatype.com/company/about}{2,000+ organizations} & \$800M (acq.) \\
\addlinespace[2pt]

5 & Semgrep & \citelink{https://semgrep.dev/}{Semgrep} & SAST (Open-Source + Cloud) & \citelink{https://semgrep.dev/}{1M+ repos scanned} & \$53M \\
\addlinespace[2pt]

6 & GitGuardian & \citelink{https://www.gitguardian.com/}{GitGuardian} & Secrets Detection & \citelink{https://www.gitguardian.com/customers}{500+ customers} & \$56M \\
\addlinespace[2pt]

7 & Socket & \citelink{https://socket.dev/}{Socket} & Supply Chain & \citelink{https://socket.dev/}{200k+ repos protected} & \$25M \\
\addlinespace[2pt]

8 & Mend (WhiteSource) & \citelink{https://www.mend.io/}{Mend} & SCA & \citelink{https://www.mend.io/customers}{1,000+ customers} & \$75M \\
\addlinespace[2pt]

9 & Vanta & \citelink{https://www.vanta.com/}{Vanta} & Compliance Auto & \citelink{https://www.vanta.com/customers}{8,000+ customers} & \$203M \\
\addlinespace[2pt]

10 & Drata & \citelink{https://drata.com/}{Drata} & Compliance Auto & \citelink{https://drata.com/customers}{5,000+ customers} & \$328M \\

\bottomrule
\end{tabular}%
}

\vspace{4pt}
\raggedright
\scriptsize
\textit{Notes:} Rankings prioritize customer counts. Snyk, Checkmarx, Veracode dominate enterprise SAST; Semgrep, Socket are dev-first; Vanta, Drata focus on compliance automation.

\end{table}

\end{frame}

\begin{frame}{Market 3.6: Security / Compliance---Analysis}
\justifying
\small
\begin{itemize}
  \item \textbf{Market Size (\$10--15B TAM):} Application security represents \textcolor{red}{\$10--15B TAM}. \citelink{https://snyk.io/}{Snyk} (\$1B funding, 3,000+ customers), \citelink{https://checkmarx.com/}{Checkmarx} (\$1.15B acquisition), and \citelink{https://www.veracode.com/}{Veracode} (\$2.5B acquisition) demonstrate this is a \textbf{proven enterprise category with multiple unicorn-scale vendors}. Compliance automation (\citelink{https://www.vanta.com/}{Vanta} 8,000+ customers, \citelink{https://drata.com/}{Drata} 5,000+ customers) is a fast-growing adjacent market.

  \item \textbf{Technical Moat (Very High):} Security has the \textbf{strongest moats in Market 3}: (1) \textbf{vulnerability databases}---Snyk, Sonatype maintain proprietary CVE intelligence updated daily; (2) \textbf{language-specific analyzers}---SAST tools require deep AST parsing for each language; (3) \textbf{supply chain intelligence}---\citelink{https://socket.dev/}{Socket} analyzes package behavior, not just known CVEs; (4) \textbf{compliance frameworks}---Vanta/Drata embed SOC2/ISO control mappings. General coding agents \textbf{cannot replicate} these specialized data assets and integrations.

  \item \textbf{AI Potential (Low--Medium---30--50\%):} Security has the \textcolor{red}{lowest AI automation ceiling} in Market 3. AI can triage alerts, suggest fixes for known CVEs, and automate compliance evidence collection. However, \textbf{false positives are costly} (alert fatigue), \textbf{false negatives are catastrophic} (missed vulnerabilities). Human security review will remain mandatory for: threat modeling, penetration testing, incident response. AI augments security teams but faces fundamental \textbf{trust barriers} that limit full automation.
\end{itemize}
\end{frame}

% =============================================================================
% Cross-Market Patterns & Outlook
% =============================================================================

\begin{frame}{Market 3: Cross-Market Patterns---Strategic Synthesis}
\justifying
\small
\begin{itemize}
  \item \textbf{Market 2 Cannibalization Risk Varies by Vertical:} Not all Market 3 verticals face equal threat from general coding agents. \textcolor{red}{High risk}: Code Review (Copilot/Cursor overlap with inline suggestions), Documentation (agents can generate docs). \textcolor{red}{Low risk}: Security (specialized data assets), Testing (execution infrastructure), DevOps (cloud integrations). The survivors will be those with \textbf{non-replicable data moats} or \textbf{infrastructure dependencies}.

  \item \textbf{The ``Workflow Layer'' Defense:} Market 3 specialists survive by owning \textbf{workflow}, not just AI capability. \citelink{https://www.sonarsource.com/}{SonarQube} gates deployments; \citelink{https://www.vanta.com/}{Vanta} automates compliance evidence; \citelink{https://spacelift.io/}{Spacelift} manages IaC state. These are \textbf{enterprise control planes} that sit outside the IDE. General agents can generate code, but specialists own the \textbf{approval, governance, and audit} layer.

  \item \textbf{AI-Native vs. AI-Augmented Incumbents:} Two strategies compete: (1) \textbf{AI-native startups} (Momentic, CodeRabbit, Kubiya) building from scratch with AI-first UX; (2) \textbf{AI-augmented incumbents} (SonarQube, Tricentis, Snyk) adding AI to existing platforms. Incumbents have distribution and data; startups have UX agility. In Testing and Security, \textbf{incumbents are likely to win} via AI bolt-ons. In Documentation and Code Review, \textbf{startups have a chance} because existing workflows are less entrenched.
\end{itemize}
\end{frame}

\begin{frame}{Market 3: 6--12 Month Outlook}
\justifying
\small
\begin{itemize}
  \item \textbf{Documentation Outlook:} Expect consolidation---\citelink{https://mintlify.com/}{Mintlify} and \citelink{https://swimm.io/}{Swimm} may merge or be acquired. \citelink{https://cognition.ai/blog/deepwiki}{DeepWiki} (via Devin) positions Cognition as documentation layer for AI agents. The question: does documentation become a \textbf{feature of Market 2 agents} (Copilot/Cursor generate docs inline) or remain a \textbf{standalone workflow}? Likely outcome: API docs specialists (ReadMe, Fern) survive; general docs tools consolidate.

  \item \textbf{Testing \& Security Outlook:} These verticals are \textbf{most defensible}. Expect Snyk, Checkmarx, Tricentis to acquire AI-native startups for capability (Momentic, Launchable). The enterprise buyer wants \textbf{consolidated platforms}, not point solutions. Security's trust barriers mean \textbf{human-in-the-loop persists}---AI augments but doesn't replace. Watch for \citelink{https://semgrep.dev/}{Semgrep} to expand from SAST into broader security platform.

  \item \textbf{Code Review \& DevOps Outlook:} Code review faces \textbf{highest cannibalization risk}---\citelink{https://marketplace.visualstudio.com/items?itemName=GitHub.copilot}{Copilot's} PR summarization and \citelink{https://cursor.com}{Cursor's} inline suggestions overlap with CodeRabbit/Qodo. Survivors will pivot to \textbf{enterprise compliance} (audit trails, policy gates). DevOps AI (\citelink{https://www.pulumi.com/product/neo/}{Pulumi Neo}, \citelink{https://www.kubiya.ai/}{Kubiya}) is early but promising---infrastructure complexity creates genuine need for AI assistance beyond code generation.
\end{itemize}
\end{frame}
