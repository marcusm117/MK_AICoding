% =============================================================================
% Part II / RQ1: Economic Impact Prediction
% =============================================================================


\section{RQ1: Economic Impact Analysis}

\begin{frame}{RQ1: Methodology}
\justifying
\textbf{Approach:} Qualitative analysis $\rightarrow$ Quantitative estimation

\vspace{6pt}
\textbf{Data Sources:}
\begin{itemize}\tightlist
  \item Public company filings, earnings calls
  \item Industry reports (Gartner, McKinsey, a16z)
  \item Academic studies on developer productivity
  \item Expert interviews and surveys
\end{itemize}

\vspace{6pt}
\textbf{Estimation Framework:}
\begin{enumerate}\tightlist
  \item Bottom-up: Task-level productivity gains $\times$ task frequency
  \item Top-down: Market size $\times$ adoption rate $\times$ efficiency factor
  \item Comparable: Historical automation impact (e.g., IDEs, DevOps)
\end{enumerate}
\mynote{Placeholder — detailed calculations to be added.}
\end{frame}


\begin{frame}{RQ1.1: Big Tech Internal Impact}
  \justifying
  \small
  \begin{itemize}\tightlist
    \item \textbf{RQ1.1.1:} How much can Big Tech (FAANG+) reduce engineering headcount while maintaining output? \\
    \textit{Quantify: \% headcount reduction or \$ saved per year per company.}
  
    \item \textbf{RQ1.1.2:} If headcount stays constant, how much faster can they deliver features/products? \\
    \textit{Quantify: \% increase in velocity (commits/PRs/features per engineer).}
  
    \item \textbf{RQ1.1.3:} Will companies hire fewer engineers but more PMs/designers? (Andrew Ng hypothesis) \\
    \textit{Quantify: Engineer:PM ratio shift (e.g., 8:1 $\rightarrow$ 5:1?).}
  
    \item \textbf{RQ1.1.4:} What is the adoption curve? Which teams/orgs adopt first? \\
    \textit{Quantify: \% of engineering orgs with >50\% AI tool penetration by 2027.}
  
    \item \textbf{RQ1.1.5:} How does AI coding affect code quality metrics (bugs, tech debt, security vulnerabilities)?
  \end{itemize}
  \end{frame}
  
  \begin{frame}{RQ1.2: Traditional Industry IT Departments}
  \justifying
  \small
  \begin{itemize}\tightlist
    \item \textbf{RQ1.2.1:} Finance sector (banks, hedge funds, trading firms): \\
    \textit{How much can they save on IT development costs? Can AI tools handle compliance-heavy codebases?}
  
    \item \textbf{RQ1.2.2:} Semiconductor/Chip Design: \\
    \textit{Can foundation models help with proprietary HDL/RTL code given minimal open-source training data? What's the productivity gain for EDA workflows?}
  
    \item \textbf{RQ1.2.3:} Manufacturing \& Industrial (automotive, aerospace, robotics): \\
    \textit{Embedded systems, safety-critical code — what's realistic adoption?}
  
    \item \textbf{RQ1.2.4:} Healthcare/Pharma IT: \\
    \textit{Regulatory constraints (HIPAA, FDA) — does AI coding help or create compliance overhead?}
  
    \item \textbf{RQ1.2.5:} Government \& Defense: \\
    \textit{Security clearance, air-gapped environments — can open-weight models capture this market?}
  \end{itemize}
  \textit{Quantify each: \$ savings, \% efficiency gain, adoption timeline.}
  \end{frame}
  
  \begin{frame}{RQ1.3: Startup Ecosystem \& VC Market}
  \justifying
  \small
  \begin{itemize}\tightlist
    \item \textbf{RQ1.3.1:} How much faster can non-AI startups iterate using AI coding tools? \\
    \textit{Quantify: Time-to-MVP reduction (weeks $\rightarrow$ days?), burn rate impact.}
  
    \item \textbf{RQ1.3.2:} Does faster iteration increase startup success rates or just failure velocity? \\
    \textit{Quantify: Expected change in startup survival rates at Series A/B.}
  
    \item \textbf{RQ1.3.3:} How does this affect VC investment thesis? \\
    \textit{Will VCs fund smaller teams? Expect faster returns? Change valuation multiples?}
  
    \item \textbf{RQ1.3.4:} Can solo founders / 2-person teams now build what required 10-person teams? \\
    \textit{Quantify: Minimum viable team size shift by company stage.}
  
    \item \textbf{RQ1.3.5:} Will AI coding tools create ``hyper-competition'' that compresses margins for all startups?
  \end{itemize}
  \end{frame}
  
  \begin{frame}{RQ1.4: Labor Market Dynamics}
  \justifying
  \small
  \begin{itemize}\tightlist
    \item \textbf{RQ1.4.1:} Net job displacement: How many SE jobs will be eliminated vs. created? \\
    \textit{Quantify: \# jobs by 2030 (pessimistic / base / optimistic scenarios).}
  
    \item \textbf{RQ1.4.2:} Which SE roles are most/least vulnerable? \\
    \textit{(Junior devs, QA, DevOps, architects, ML engineers — rank by displacement risk.)}
  
    \item \textbf{RQ1.4.3:} Wage effects: Will AI tools compress or polarize SE salaries? \\
    \textit{Quantify: Expected wage change by experience level and role.}
  
    \item \textbf{RQ1.4.4:} Geographic redistribution: Will AI tools accelerate offshoring or re-shoring? \\
    \textit{(If AI handles routine work, does location matter less or more?)}
  
    \item \textbf{RQ1.4.5:} New roles created: AI-assisted code reviewers, prompt engineers, AI ops? \\
    \textit{Quantify: Projected job openings in new categories.}
  \end{itemize}
  \end{frame}
  
  \begin{frame}{RQ1.5 \& RQ1.6: Macro Effects \& Risks}
  \justifying
  \small
  \textbf{RQ1.5: Macro-Economic Spillovers}
  \begin{itemize}\tightlist
    \item \textbf{RQ1.5.1:} Contribution to GDP growth from software productivity gains?
    \item \textbf{RQ1.5.2:} Will we see a ``productivity paradox'' (Solow) — gains visible in tools but not GDP?
    \item \textbf{RQ1.5.3:} Impact on software-intensive industries beyond tech (logistics, retail, media)?
  \end{itemize}
  
  \vspace{6pt}
  \textbf{RQ1.6: Negative Externalities \& Risks}
  \begin{itemize}\tightlist
    \item \textbf{RQ1.6.1:} Security risks: AI-generated vulnerabilities, supply chain attacks?
    \item \textbf{RQ1.6.2:} Technical debt accumulation: Does AI code create hidden maintenance costs?
    \item \textbf{RQ1.6.3:} IP/licensing risks: Who owns AI-generated code? Training data lawsuits?
    \item \textbf{RQ1.6.4:} Skill atrophy: Will engineers lose fundamental skills by over-relying on AI?
    \item \textbf{RQ1.6.5:} Concentration risk: What if dominant AI coding tools have outages/go out of business?
  \end{itemize}
  \end{frame}
  
% -----------------------------------------------------------------------------
% RQ1.1-RQ1.6 detailed analysis to be added
% -----------------------------------------------------------------------------

% \begin{frame}{RQ1.1: Big Tech Impact — Detailed Analysis}
% \end{frame}

% \begin{frame}{RQ1.2: Traditional Industry Impact — Detailed Analysis}
% \end{frame}

% etc.
